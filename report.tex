% 環境設定
% !TEX program = xelatex
% !TEX encoding = UTF-8
% !BIB program = biber
\documentclass[12pt,a4paper]{article}

% 字體套件
\usepackage{fontspec}
\usepackage[BoldFont]{xeCJK}
% 字體設定
\setCJKmainfont{DFKai-SB}     % 標楷體
\setmainfont{Times New Roman}
% 中文自動換行
\XeTeXlinebreaklocale "zh"             
\XeTeXlinebreakskip = 0pt plus 1pt     

% 紙張格式
\usepackage{geometry}
\geometry{
 a4paper,
 left=12.7mm,       % 左邊界
 right=12.7mm,      % 右邊界
 top=12.7mm,        % 上邊界
 bottom=12.7mm,     % 下邊界
}

% 套件
\usepackage{amssymb}
\usepackage{mathtools}
\usepackage{microtype}
\usepackage{enumitem}
\usepackage{dirtytalk}
\usepackage[figurename=圖]{caption}
\usepackage{subcaption}
\usepackage{fancyhdr}
\usepackage{graphicx}
\usepackage{hyperref}

% 設定
\renewcommand{\tablename}{表} % 改變表的名稱
\pagestyle{fancy}
\fancyhead{} % 清除所有頁首設定
\fancyfoot{}
\renewcommand{\headrulewidth}{0pt}
\renewcommand{\footrulewidth}{0pt}
\setlength{\headwidth}{\textwidth}

\lfoot{表 CM03}
\rfoot{共~~\pageref{LastPage}~~頁~~第{~~\thepage~~}頁}

\begin{document}

\noindent{\bf \large 三、研究計畫內容(以中文或英文撰寫):} 

\begin{enumerate}
\item[(一)] 研究計畫之背景。請詳述本研究計畫所要探討或解決的問題、研究原創性、重要性、預期影響性及國內外有關本計畫之研究情況、重要參考文獻之評述等。如為連續性計畫應說明上年度研究進度。
\item[] 1. 研究計畫之背景
\item[] 2. 研究問題
\item[] 3. 研究原創性
\item[] 4. 研究重要性
\item[] 5. 預期影響性
\item[] 6. 國內外研究情況
\item[] 7. 重要參考文獻


\item[(二)] 研究方法、進行步驟及執行進度。請分年列述:1. 本計畫採用之研究方法與原因及其創新性。2. 預計可能遭遇之困難及解決途徑。3. 重要儀器之配合使用情形。4. 如為須赴國外或大陸地區研究,請詳述其必要性以及預期效益等。
\item[] 1.研究方法
\item[] 2.研究步驟
\item[] 3.執行進度
\item[] 4.困難與解決途徑
\item[] 5.重要儀器之配合使用情形

\item[(三)] 預期完成之工作項目及成果。請分年列述:1. 預期完成之工作項目。2. 對於參與之工作人員,預期可獲之訓練。3. 預期完成之研究成果(如實務應用績效、期刊論文、研討會論文、專書、技術報告、專利或技術移轉等質與量之預期成果)。4. 學術研究、國家發展及其他應用方面預期之貢獻。
\item[] 1. 預期完成之工作項目
\item[] 2. 對於參與之工作人員,預期可獲之訓練
\item[] 3. 預期完成之研究成果
\item[] 4. 學術研究、國家發展及其他應用方面預期之貢獻

% \item[(四)] 整合型研究計畫說明。如為整合型研究計畫請就以上各點分別說明與其他子計畫之相關性。
\end{enumerate}	

\label{LastPage}
\end{document}
