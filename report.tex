% 環境設定
% !TEX program = xelatex
% !TEX encoding = UTF-8
% !BIB program = biber
\documentclass[12pt,a4paper]{article}

% 字體套件
\usepackage{fontspec}
\usepackage[BoldFont]{xeCJK}
% 字體設定
\setCJKmainfont{DFKai-SB}     % 標楷體
\setmainfont{Times New Roman}
% 中文自動換行
\XeTeXlinebreaklocale "zh"             
\XeTeXlinebreakskip = 0pt plus 1pt     

% 紙張格式
\usepackage{geometry}
\geometry{
 a4paper,
 left=12.7mm,       % 左邊界
 right=12.7mm,      % 右邊界
 top=12.7mm,        % 上邊界
 bottom=12.7mm,     % 下邊界
}

% 套件
\usepackage{amssymb}
\usepackage{mathtools}
\usepackage{microtype}
\usepackage{enumitem}
\usepackage{dirtytalk}
\usepackage[figurename=圖]{caption}
\usepackage{subcaption}
\usepackage{fancyhdr}
\usepackage{graphicx}
\usepackage{hyperref}

% 設定
\renewcommand{\tablename}{表} % 改變表的名稱
\pagestyle{fancy}
\fancyhead{} % 清除所有頁首設定
\fancyfoot{}
\renewcommand{\headrulewidth}{0pt}
\renewcommand{\footrulewidth}{0pt}
\setlength{\headwidth}{\textwidth}

\lfoot{表 CM03}
\rfoot{共~~\pageref{LastPage}~~頁~~第{~~\thepage~~}頁}

\begin{document}

\noindent{\bf \large 三、研究計畫內容(以中文或英文撰寫):} 

\begin{enumerate}
\item[(一)] 研究計畫之背景。請詳述本研究計畫所要探討或解決的問題、研究原創性、重要性、預期影響性及國內外有關本計畫之研究情況、重要參考文獻之評述等。如為連續性計畫應說明上年度研究進度。
\begin{enumerate}[label=\arabic*.]
\item 研究計畫之背景\\
  本研究旨在開發並實現具機器學習(Machine Learning, ML)功能的智慧化水質分析與評估系統,以解決目前水質資料分析缺乏高效率方法的問題。研究作者曾擔任科技部計畫兼任研究助理,協助執行「運用智慧型綠能永續景觀噴泉促進環境保育效益之研究」(計畫編號:MOST 110-2410-H025-024),該計畫運用物聯網(Internet of Things, IoT)技術蒐集南投麒麟潭的水質資料。藉由該計畫經驗發現自動化運行的結果相較於人工記錄更有效率,但後續資料分析仍需耗費大量工作時間。
\item 研究問題\\
  在水質資料分析方面,除了使用Microsoft Excel直接計算外,通常會使用如Statistics Analysis System(SAS)或Statistical Product and Service Solutions(SPSS)等統計專有軟體(Proprietary Software)。然而,這些軟體為專有(Proprietary)且非開源,不僅需要額外購買授權,還需投入時間學習才能熟練使用。由於其操作高度依賴使用者的專業知識,分析結果可能因個人經驗而有所差異。由於這些軟體無法修改與物聯網設備直接進行資料串聯傳輸,造成未來計畫整合實現自動化資料收集與資料分析的障礙。
\item 研究原創性\\
  近十年來,人工智慧(Artifical Intellgience, AI)從理論發展到實際應用,各種AI等工具也被廣泛運用於日常問題成為解決方案(Shao et al., 2021)。本研究運用人工智慧中的機器學習技術,開發一套智慧化水質分析與評估系統,實現更高效且易於操作的創新水質分析方法。不同於傳統統計軟體需要人工輸入與分析,本系統將結合自動化資料處理與機器學習演算法,以提升水質分析的準確性與效率。此外,本研究將探索AI在水資源管理與環境保育領域的創新應用,提供創新的智慧化水質資料分析方案。\\  
  為了找出最適合水質分析的機器學習模型,本研究針對XGBoost、LightGBM、支援向量機(Support Vector Machine, SVM)、隨機森林(Random Forest, RF)、多元多項式迴歸(Multiple Polynomial Regression, MPR)以及線性迴歸(Linear Regression, LR)等多種模型進行比較與評估。透過實驗分析與比較不同模型在水質資料分析準確度、計算效能與資料適應性的表現,進一步應用於改良水質分析,使系統能夠有效應對水體環境的分析與評估需求。\\
  實驗設計包括使用先前於「運用智慧型綠能永續景觀噴泉促進環境保育效益之研究」計畫中,使用物聯網設備實際蒐集的水質資料,加上環境部環境資料開放平臺公開資料使用,特徵選擇包括溶解氧(DO)、生化需氧量(BOD)、氨氮(NH3-N)、電導度(EC)和懸浮固體(SS)等關鍵指標。在訓練過程中,使用水體品質指標(Water Quality Index, WQI$_5$)作為標記資料進行監督式學習。
\item 研究重要性\\
  在過去十年內,人工智慧的研究從長期被忽視的陰霾中走了出來,被人們看見了希望且掀起了新一波風潮(Cowls, 2021),而水資源管理與環境生態保育是全球關注的議題,而水質分析與評估是確保水資源安全的重要環節。本研究透過機器學習技術,克服傳統方法的效率瓶頸,降低對專業知識依賴,使水質資料分析變得更有效且易於操作,提升水質資料分析技術的普及性與可行性。\\  
  本研究針對XGBoost、LightGBM、支援向量機(SVM)、隨機森林(RF)、多元多項式迴歸(MPR)以及線性迴歸(LR)等機器學習模型進行獨立訓練,並透過決定係數($R^2$)、均方根誤差(Root Mean Square Error, RMSE)、平均絕對誤差(Mean Absolute Error, MAE)以及殘差分析等指標評估各模型在水質資料分析上的準確性。\\  
  透過實驗分析,我們發現[TODO]。這些結果有助於選擇最適合水質資料分析的機器學習模型,進一步提升水資源管理的準確度,為環境監測技術與未來相關研究的發展提供有效的參考依據。
\item 預期影響性\\
  本研究的智慧化水質分析與評估系統結合了模型比較實驗與系統開發,有望在以下方面產生影響:
    \begin{itemize}
    \item 技術創新:人工智慧未來將成為人類不可或缺的夥伴(Aggarwal et al., 2022),本研究在水質資料分析引入機器學習技術,提升分析準確性與自動化程度。
    \item 環境保育:能夠在水資源管理方面準確且有效地發現水質變化,改善水體生態環境。
    \item 跨領域應用:本研究結合了資訊工程、人工智慧、水資源管理和環境生態保育等不同領域,未來也可擴展應用於不同環境監測領域,例如空氣品質監測或污染物分析,促進人工智慧在環境科學中的應用。
    \end{itemize}
\item 國內外研究情況\\
  本研究作者先前參與的計畫已經使用物聯網技術進行水質資料的採集,這種方法在環境監測中已經相當普遍。然而,目前國內外針對水質資料分析的智慧化系統,特別是結合機器學習技術的水質分析系統相對較少。機器學習技術在水質評估中的應用已被證明是有效的,並逐漸成為水資源管理的重要分析工具(Zhou et al., 2023),而Im et al.(2022)應用了各種技術構建學習模型,可以提供即時準確水質資訊,並通過早期診斷水質異常以改善公共健康。因此,本研究具有創新性和獨特性,有助於填補水質資料分析中高效方法的空白。
\item 重要參考文獻\\
  
\end{enumerate}

\item[(二)] 研究方法、進行步驟及執行進度。請分年列述:1. 本計畫採用之研究方法與原因及其創新性。2. 預計可能遭遇之困難及解決途徑。3. 重要儀器之配合使用情形。4. 如為須赴國外或大陸地區研究,請詳述其必要性以及預期效益等。
\begin{enumerate}[label=\arabic*.]
\item 研究方法\\
\item 研究步驟\\
\item 執行進度\\
\item 預計可能遭遇之困難與解決途徑\\
\item 重要儀器之配合使用情形\\
\end{enumerate}

\item[(三)] 預期完成之工作項目及成果。請分年列述:1. 預期完成之工作項目。2. 對於參與之工作人員,預期可獲之訓練。3. 預期完成之研究成果(如實務應用績效、期刊論文、研討會論文、專書、技術報告、專利或技術移轉等質與量之預期成果)。4. 學術研究、國家發展及其他應用方面預期之貢獻。
\begin{enumerate}[label=\arabic*.]
\item 預期完成之工作項目\\
\item 對於參與之工作人員,預期可獲之訓練\\
\item 預期完成之研究成果\\
\item 學術研究、國家發展及其他應用方面預期之貢獻\\
\end{enumerate}

\end{enumerate}	

\label{LastPage}
\end{document}
